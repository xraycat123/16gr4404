\newpage
\section{Hjerteudredning}
Sygdomme der optræder optræder i hjerte og/eller blodkar har fælles betegnelsen hjertekarsygdomme. Af sygdomme i hjertet kan der f.eks. nævnes hjerteklapfejl, hjertesvigt og forkammerflimmer. Åreforkalkning er en karsygdom der kan give indsnævring i blodårene og dermed reduceret blodgennemstrømning. Dette kan lede til en blodprop, der i princippet kan opstå overalt i kroppen, f.eks. i hjertet, hjernen, benene eller lungerne. %Ved skade på det elektriske system kan det resultere i en unormal hjerterytme eller abnormale kontraktioner af hjertet der medfører en svækkelse af hjertes pumpefunktion \cite{guyton}.
\todo{se det der udkommenteret}
Symptomerne ved hjertekarsygdomme afhænger af det berørte organ, i nogle tilfælde kan patienten dog være symptomfri. Det kræver derfor en grundig lægeundersøgelse for at stille diagnose og bestemme behandlingsbehov \cite{apoteket}. 

Ved den praktiserende læge er hjerteudredning udført på bagrund af en objektiv undersøgelse og patientens sygdomshistorik. Nogle gange vælger lægen at udføre en længerevarende monitorering, hvor patienten får et måleapparat med hjem. Hvis den praktiserende læge vurder at der begrundet mistanke om en hjertesygdom, vil lægen henvise til en hjertespecialist. Afhængigt af symptomerne anvendes der herefter en række forskellige laboratorieteknikker til diagnosticering. I akutte tilfælde bliver patienten dog indlagt med det samme \cite{hjerud}. Et overblik over diagnoseringsforløbet kan ses på figur \ref{fig:forloeb}, og underpunkterne bliver nærmere forklaret i \todo{næste afsnit/under afsnit istedet for kommende afsnit} kommende afsnit.

\begin{figure}[H] % Example of including images
\begin{center}
\includegraphics[width=1\textwidth]{figures/forloeb}
\end{center}
\caption{Overblik over diagnoseringsforløbet af hjertekarsygdomme.}
\label{fig:forloeb}
\end{figure}

\subsection{Objektiv undersøgelse}
Ved tidlig mistanke om et kardiovaskulært problem, kan lægen anvende et stetoskop til at lytte på hjertet. Hjertelydene identificeres efter rytme, første- og anden hjertelyd, systole og diastole samt mislyde. Disse giver lægen mulighed for at identificere hvilken del af hjertet der er eventuelle komplikationer med. Ved måling med et stetoskop skal lægen manuelt lytte på patientens hjerte og selv vurdere om eventuelle hjertefejl, hvilket kræver at lægen skal have nogen erfaring med dette \cite{subob}. Herudover vil lægen spørge patienten ind til symptomerne, risikofaktorer og eventuelt sygdomshistorikken i vedkommendes familien \cite{hjerud}. 
 
\subsection{Laboratorieundersøgelse}
I laboratoriet undersøger specialisten hjertets elektriske aktivitet (som blev beskrevet i afsnit \ref{Hjertets_elektriske_ledningssystem}) vha. en elektrokardiografi (EKG), hvilket giver udtryk for i hvilken grad hjertets evne til at slå er intakt \cite{ekg}. EKG-signalet giver et billede af hjertets evne til at lede elektrisk strøm til hjertemuskulaturen, men dog ikke i hvor effektivt hjertet er til at flytte blodet. Ved at måle EKG kan lægen dermed finde ud af om en eventuelt fejl på hjertet skyldes problemer med den elektriske ledning \cite{ekg}.
 
Hvis komplikationerne vedrørende hjertet ikke er elektrisk relaterede, giver anledning på at kigge på den mekaniske aktivitet. Dette kan gøres ved at kigge på hjertets blodtilførelse undersøges gennem et ultralydsapparat – denne metode kaldes ekkokardiografi (EKKO). Undersøgelsen giver information om, hvordan hjertet er bygget op, og hvordan de enkelte dele af hjertet fungerer. EKKO kan afsløre forskellige sygdomme og tilstande ved en direkte fremstilling af alle hjertets dele, som aflæses på en skærm \cite{hjerud}.
 
For at kortlægge koronararterierne, bliver der ofte anvendt koronarangiografi (KAG). Metoden kræver at lægen fører et kateter ind i patientens koronararterier, hvorefter en kontrastvæske indsprøjtes, der gør arterierne synlige på røntgenbilleder. KAG kan påvise forskellige grader af forkalkninger og forsnævringer i koronararterierne \cite{hjerud}. 
 
\subsection{Hjemmemonitorering}
Hjemmemonitorering er en type af telemedicin, der muliggør monitorering af patienter uden for konventionelle kliniske indstillinger (f.eks. i hjemmet). I 2012 lancerede Regeringen, Kommuners Landsforening (KL) og Danske regioner en handlingsplan for udbredelsen af telemedicin. I denne er der forklaret at telemedicin kan have en række potentielle fordele både for patienten, økonomien og den sundhedsfaglige medarbejder, så som:
\begin{itemize}
\item Tættere koordinering mellem den praktiserende læge, den kommunale pleje og sygehusene giver øget kvalitet og sikkerhed i behandlingen.
\item Patienten får indsigt i sin sygdom og bedre muligheder for at deltage aktivt i sin behandling.
\item Mere fleksible og effektive måder at organisere arbejdet i sundhedsvæsenet giver færre sygehusindlæggelser, ambulante kontroller og hjemmeplejebesøg.
\end{itemize}
 
\noindent I forhold til hjerteudredning har patienten mulighed for at få udleveret hjemmemonitorerings udstyr, for at måle enten blodtrykket eller EKG. Ved mistanke om hypo- eller hypertension bærer patienten et blodtryksapparat i et døgn, hvor blodtrykket bliver målt flere gange i løbet af denne periode. Tilsvarende, hvis der er mistanke om hjerterytmeforstyrrelser anvendes Holtermonitorering. I dette tilfælde får patienten påsat elektroder og en lille båndoptager, som optager EKG-signal i op til flere døgn \cite{hjerud}.
 
Udover blodtryksmålere, er der mangel på en længerevarende monitorering der giver information om hjertets mekaniske aktivitet. Med hensyn til alle gevinsterne ved hjemmemonitoreringssystemer kunne sådan et system med fordel tilføjes i sundhedssektoren.