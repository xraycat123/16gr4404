\section{Diagnosticering af hjerteproblemer}

I Danmark er der ifølge \ref{Indledning} \todo{Skal fikses så den passer med navnet} en stor udbredelse af hjerte-karsygdomme. Ved mistanke om lidelser i hjertet er det muligt at konsultere en læge, for at få kortlagt symptomerne. Lægen laver en hjerteudredning der bygger på undersøgelser og tidligere sygdomshistorik \cite{hjerud}.

\subsection{Anamnese}
Ved tidlig mistanke om et kardiovaskulært problem, kan den praktiserende læge anvende et stetoskop til at lytte på hjertet \cite{subob}. Hjertelydende identificeres efter rytme, første- og anden hjertelyd, systole og diastole samt mislyde. Disse giver lægen mulighed for at identificere hvilken del af hjertet der er eventuelle komplikationer med. Målingen med stetoskop kræver at lægen manuelt lytter på patientens hjerte via et stetoskop, hvilket kræver at lægen skal trænes til at være i stand til dette, hvilket tager tid for lægen og samtidig giver mulighed for fejldiagnosticeringer. 

\subsection{Laboratorie undersøgelse}
I laboratoriet undersøger lægen hjertets evne til at generere og lede elektrisk strøm via en EKG måler (Elektrokardiografi). EKG signalet giver udtryk for i hvilken grad hjertets evne til at slå er intakt. Dette måles udfra hjertets elektriske aktivitet over længere tid \cite{ekg}. Se afsnit \ref{Hjertets_elektriske_ledningssystem}\todo{Label skal passe.}. Et eksempel på et EKG signal kan ses på \ref{ekgbil1}\todo{Skal finde et godt billede}. 

\label{ekgbil1}
%\includegraphics[]{}
\ref{ekgbil1} viser et EKG signal på en rask patient. Signalet giver udtryk for stimulering af muskelvævet i hjertet. 

EKG signalet giver et billede af hjertets evne til at lede elektrisk strøm til hjertemuskulaturen, men dog ikke i hvor effektivt hjertet er til at flytte blodet. Dette er fordi EKG ikke er et mekanisk signal, men et elektrisk signal. Ved at måle EKG kan lægen dermed finde ud af om fejlen på hjertet skyldes problemer med den elektriske ledning, eller om problemet omhandler f.eks. muskelvæv eller hjertets klapper \cite{ekg}. 

Lægen har mulighed for at udlevere hjemmemonitorerings udstyr til patienten, ved mistanke om for højt eller for lavt blodtryk \cite{hjerud}. Ved at anvende hjemmemonitorering, behøver patienten ikke monitoreres på hospitalet. Dette er uddybet i afsnit \ref{telemedicin}\todo{Skal sikre at labellen passer}.

Hvis komplikationerne omkring hjertet ikke er elektriske relaterede, kan men ved ekkokardiografi undersøge hjertets blodtilførelse. Dette undersøges gennem et ultralydsapparat.



\subsubsection{Invasiv undersøgelse}
For at undersøge hjertets tilføring af blod, kan hjertet også undersøges invasivt.
Metoden kræver at lægen fører et kateter ind i patientens koronararterierne, hvorefter en kontrastvæske indsprøjtes \cite{hjerud}. Derefter undersøges hjertet ved brug af røntgenbilleder. Denne metode er invasiv, og giver et stilbillede af personens hjerte. 


\subsection{White coat hypertension}
White coat hypertension kan forekomme hos enkelte personer når de er ved lægen, eller på et hospital \cite{wch}. Det forhøjede blodtryk kan skyldes forhøjet puls, og kan derfor give et misvisende billede af patientens hjertes tilstand. Det er dog uklart i hvilken grad pulsen påvirkes, men ældre mennesker, og især kvinder har højere puls ved måling på hospital, frem for i eget hjem  \cite{bpm}. Da normalpraksis for blodtryksmåling og pulsmåling er at blive målt på af en læge, kan der være brug for et alternativ. Dette kan være i form at et hjemmemonitorerings system med samme egenskaber som udstyret der anvendes ved lægen. Dermed kan kan fejl målingerne der forekommer ved lægen undgås. I et australsk review af white coat hypertension fra 2015 \cite{wch}, estimeres det at cirka 15\% af populationen har denne lidelse. I samme review anbefales det også at anvende hjemmemonitorerings systemer og giver et lige så godt billede af patientens helbredstilstand, som det ambulatoriske udstyr. 



