\section{SCG}\label{scg}
\subsection {Hvad er fordelene og ulemperne ved at bruge det?}

\begin{itemize} 
	\item Mekanisk aktivitet 

	\item Non-invasiv 

		\begin{itemize}
			\item Nemt at sætte op til hjemmemonitorering. 
		\end{itemize}

	\item Usikkerhed ifht. tolkningen af signalet 

	\item Metoden er ikke særlig dokumenteret.
\end{itemize}

SCG måler den mekaniske aktivitet af hjertet, via vibrationer gennem thorax. Metoden er kendt under flere betegnelser, alt afhængigt at positionen; apex-cardiografi, kineto-cardiografi eller ballistocardiografi \cite{Wearable}. Med fremskridtet inden for accelerometre, er størrelsen på disse skrumpet betydeligt siden de første kliniske forsøg i midten af halvfemserne, hvilket muliggør brugen af disse i praksis. SCG er af ikke invasiv karakter, hvorfor påmontering i eget hjem med lethed ville kunne implementeres, hvad enten der er tales om patches eller straps til påmontering. Et studie af \cite{Wearable} har implementeret et accelerometer i en tætsiddende trøje for lettere at kunne forbinde patienten til sensoren.


Med let tilgængelighed samt håndterbarheden af moderne accelerometre \todo{giv eksempel på den model vi benytter} er der muligheder for monitorering af SCG enheder på patienter i hjemmet, med telemedicinsk opkobling til hospitalet. Således vil lægefagligt personale løbende kunne monitorere patienter i risikogruppe for kardiovaskulære lidelser. Med indbygget trådløs opkobling til smartphones eller lignende anordninger vil omkostningerne på implementering kunne nedbringes. Da det mindsker de krav der stille den mængde af hardware der skal stilles til rådighed/udvikles af sundhedssektoren.


SCG signaler optages ofte med et enkelt akset eller 3-akset accelerometer, størstedelen af den fundne litteratur beskriver brugen af et enkelt akset accelerometer der fokuserer på bevægelser i dorso–ventral retning. Fælles for begge optage muligheder er at de optagne signaler kan forstyrres af; sensor, kredsløbsstøj, bevægelses artefakter, underlags vibrationer mv. Det er derfor nødvendigt med grundig filtrering af signalet \cite{Recent_Advances}.


SCG signalet indeholder lavfrekvent information er det derfor modtageligt overfor “pink støj” (1/f). SCG målt på ældre mennesker viser en lavere signal amplitude i forhold til raske unge mennesker \cite{Recent_Advances}. Placering af måleinstrumentet er ligeledes af stor betydning, da signaler er direkte påvirker af placeringen af SCG sensoren.


Der findes grundet det sene gennembrud med teknologien en begrænset mængde litteratur på SCG området.

\subsection{Der mangler:} 
\begin{itemize}
	\item Noget med modelleringsarbejde
	\item Standard modeller. Usikkerhed ifht. tolkningen af signalet
\end{itemize}