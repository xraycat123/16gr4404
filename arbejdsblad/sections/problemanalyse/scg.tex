\section{SCG}
Flere metoder kan benyttes til at at måle de vibrationer forårsaget af det kardiovaskulæresystem. Her kan f.eks. nævnes ballistocardiografi, mechanocardigram og seismocardiografi. Disse vibrationer indeholder information om den mekaniske funktion af det kardiovaskulære system. En metode der er blevet foreslået til at være et brugbart værktøj ifht. detektering af begivenhederne i hjertets cyklus er SCG \cite{phd}. 

Kort beskrevet repræsenterer SCG-signalet de vibrationer i brystvæggen, der er forårsaget af hjertets sammentrækning og udstrømning af blod fra ventriklerne. Ved hjælp af  SCG kan den mekaniske aktivitet af hjertet måles, via vibrationer på brystvæggen. Disse vibrationer optages ofte med et enkelt akset eller 3-akset accelerometer, dog beskriver størstedelen af den fundne litteratur  brugen af et enkelt akset accelerometer der fokuserer på bevægelser i den dorso–ventrale retning \cite{Recent_Advances}.

\section{SCG-signalet}
Acceleration er den tidsafledede af hastigheden, altså tilvækst i hastighed pr. tidsendhed. Derfor giver SCG god information om hjertes bevægelser igennem dens cyklus \cite{performance}, dog  foreligger en variation i tolkningen af betydningen af de forskellige events i SCG-signalet \cite{phd}. Udfordringerne kommer af den komplekse bølgeform som er vanskelig at fortolke \cite{zanetti}, da man ikke fuld forståelse for hvordan alle begivenheder i hjertecyklussen giver sig til kende på SCG-signalet, da bølgerne fra de enkelte begivenheder kan påvirke hinanden \cite{abra}.

Størstdelen af SCG signal befinder sig området 4-50 Hz, hvoraf hovedparten af signalet vil ligge under 20 Hz i de fleste tilfælde \cite{phd}. 
 SCG målt på ældre mennesker viser en lavere signal amplitude i forhold til raske unge mennesker \cite{Recent_Advances}. Placering af måleinstrumentet er ligeledes af stor betydning, da signaler er direkte påvirker af placeringen af SCG sensoren \cite{zanetti}. SCG vil desuden oftes måles sammen med EKG for at kunne aflæse de elektromekaniske egenskaber for hjertet \cite{abra}. De forskellige begiveheder er illustreret på  \ref{fig:wigger}

\begin{figure}\label{fig:wigger}
\includegraphics[scale=1]{figures/wigger.PNG}
\caption{De forskellige begivenheder i SCG signalet. Kilde: \citep{zanetti}}
\end{figure}

\section{Anvendelse af SCG}
SCG har været foreslået til at måling af en række parametre. Der kan bl.a. nævnes, at har været foreslået til at kunne tilnærme sig myocardielle kontratillitet udtrykt ved dP/dt_max (maksimum afledte af tryk). Den nuværende guldstandard for måling af samme paramteter foregår ved benyttelse af et kateter, hvor trykket i ventre ventrikel måles. SCG har også været foreslået til at estimere hjertets slagvolumen. Slagvolumen er en indikator for den myocardiale kontraktilitet og har en tæt korrelation til dP/dt_max..SCG er også blevet foreslået til måle begivenhederne i de systoliske og diastoliske tidsintervaller for at estimere ydeevnen for den venstre ventrikel. Her kan nævnes  præ-ejektions perioden (PEP) og venstre ventrikulær ejections tid (LVET) kan aflæses ud SCG-signalet, hvilket også kan ses på figur \ref{wiggers}. Forlænget PEP og forkortet LVET er relateret til reduceret slagvolumen og cardiac output. Desuden er det blevet vist at PEP forkortes når kontraktiliteten stiger, hvilket f.eks. kan benyttes som præ-screening af patienter med synkroniseringsproblemer i hjertet  \cite{Zanetti}  \cite{abra}. Af andre anvendelser af SCG kan f.eks. nævnes diagnose og monitering af koronararteriesygdom og måling af heart rate varibilty.

Majoriteten af studier omhandlende SCG, fokuseres der ofte på optage informationer der relaterer sig til det kardiovaskulære system. Flere studier har dog foreslået at SCG er velegnet til at udtrække informationer om respiration. Studiet fandt at start-ekspiration og peak-ekspiration kunne udledes fra SCG signalet \cite{pandia} \cite{magic}. Denne information vil  bl.a. kunne bruges til detektere abnormaliteter i respirationsmønster ved søvnforstyrrelser \cite{tavaloka}.  Respiration kan påvirke signalet ved enten at påvirke SCG signalets baseline, ændre amplituden af bølgerne i SCG-signalet og ændring i RRI (low-frequency power of R-R interval). Hvor meget man kan aflæse ud fra de enkelt parametre er dog meget individuelt \cite{magic}.