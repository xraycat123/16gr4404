\section{Problemformulering}

En række forskellige sygdomme kan påvirke hjertet og blodkar, og ligeledes er der mange  symptomer relaterede til disse. Derfor kræver det ofte en grundig lægeundersøgelse for at finde ud af om, og i så fald, hvilken sygdom en person har. Det nuværende diagnosticeringsforløb kan inddeles i tre faser. Den første er objektiv undersøgelse ved den praktiserende læge, hvor bl.a. patientes sygdomshistorik og riskiofaktorer er taget i betragtning. Derefter kan patienten få henvisning til en hjertespecialist, hvor der bliver anvendt forskellige laboratorieundersøgelser afhængigt at symptomerne. Herunder kan nævnes EKG til undersøgelse af hjertets elektriske aktivtiet, EKKO der giver information om hjertets opbygning, og KAG til undersøgelse af koronararterierne. Nogle gange er der mulighed for observertion vha. hjemmemonitorering, inden patienten går til en hjertespecialist. Dette kan enten være måling af blodtryk eller EKG og har til formål at indsamle mere data over de fysiologiske ændringer i det kardiovaskulære system, med henblik på at opdage abnormaliteter. Typisk vil hjemmemonitorering til hjerteudredning vare i et til to døgn. Udover en blodtryksmåler er der ikke nogle monitoreringsværktøjer anvendte på nuværende tidspunkt, der giver udtryk for den mekaniske aktivitet i hjertet. Sådan et system kunne med fordel tilføjes udredningforløbet af hjertekarsygdomme. 

En metode der kan anvedes til måling af hjertets mekaniske funktion er SCG, der respræsenterer vibrationer på brystvæggen. SCG er af ikke invasiv karakter, og vil oftest måles sammen med EKG for at kunne aflæse de samlede elektromekaniske egenskaber for hjertet. SCG har været foreslået til måling af bl.a. myocardielle kontraktilitet, hjertets slagvolumen og  de systoliske og diastoliske tidsintervaller. Desuden kan respiration påvirke SCG signalets baseline eller amplitude, og dermed kan information om dette også udtrækkes fra SCG signalet. Disse parametre kan bruges til screening og diagnosering af forskellige hjertekarsygdomme, hvorfor SCG er en spændende mulighed der kunne bruges til et hjemmemonitoreringssystem i forbindelse med hjerteudredning.  Dette leder frem til følgende problemformulering: 


\begin{center}\textit{Hvordan kan et system designes, implementeres og testes således, at SCG-signalet kan  analyseres, så det kan benyttes til at kortlægge de forskellige begivenheder i hjertets cyklus, og hvilke påvirkninger vil respiration have på SCG-signalet?}\end{center}