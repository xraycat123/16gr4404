\section{Telemedicin}
Telemedicin kan defineres som tid-, sted- og rumuafhængige digitalt understøttede sundhedsydelser, leveret over afstand med potentiale til at skabe målbar sundhedsmæssig gevinst eller værdi\cite{Regioner2012}.

På grund af den demografiske udvikling, det økonomiske pres på den offentlige sektor i de kommende år, er der behov for at sikre bedre ressourceudnyttelse. Telemedicinske løsninger kan bidrage hertil.\cite{Regioner2012}
 
Danmark er et af de førende europæiske lande inden for telemedicin og formåer blandt andet at gå fra pilot projekter til at indføre telemedicin i normal praksis.\cite{FabienneAbadieCristianoCodagnone2011}.
  

\subsection{Telemedicin - Danmark}

I 2012 lanceredes: \textquotedbl National handlingsplan for udbredelse af telemedicin\textquotedbl. Handlingsplanen er kreeret af regeringen, KL og Danske regioner i fællesskab med fokus på det tværorganisationelle samarbejde Handlingsplanen løb fra 2012 til 2015. 
Handlingsplanen overordnet formål bidrager til, at Danmark også fremadrettet er blandt frontløberne, når det gælder bud på, hvordan ny teknologi og nye arbejdsgange kan løfte de demografiske og økonomiske udfordringer.\cite{Regioner2012} 
I den telemedicinske handlingsplan afprøves en 5 Klinisk Integrerede telemedicinske Hjemmemonitoreringsrojekter med 1.126 deltagene borgere. Alle forsøg omhandlede monitorering i eget hjem med målinger og elektronisk overførsel af data samt video eller anden form for elektronisk kontakt. Handlingsplanen omhandler 3 hovedpunkter: \cite{Regeringen2012} 

\begin{itemize} 
	\item Udgangspunktet: Store udfordringer – Nye muligheder
		\begin{itemize} 

			\item nye arbejdsgange og telemedicinske metoder skal dæmpe presset på ressourcerne i sundhedssektoren. Ressourcerne vil stige med henblik på en stigende del af ældre, op imod 84\% i 2040, nye behandlingsmetoder og flere kroniske syge patienter.

		\end{itemize} 

	\item Bedre rammer for telemedicin
		\begin{itemize} 

			\item Fælles nationale standarder for sundheds-It skal sammen med evaluering af projekter gøre kommunerne og regionernen i stand til at udnytte telemedicin.

		\end{itemize} 

	\item Veje til udbredelse
		\begin{itemize} 

			\item Telemedicinske projekter skal evalueres og de bedste løsninger skal udbredes. Handlingsplanen tager udgangspunkt i 5 konkrete projekter

				\begin{itemize} 

					\item Klinisk integreret hjemmemonitorering.
					\item Hjemmemonitorering for KOL-patienter.
					\item Telepsykiatri.
					\item Internetpsykiatri.
					\item National udbredelse af telemedicinsk sårvurdering.

				\end{itemize} 

		\end{itemize} 

		
\end{itemize} 


\subsubsection{Fordele}

Telemedicin kan have en række potentielle fordele både for patienten, økonomien og den sundhedsfaglige medarbejder. Ifølge \textquotedbl National handlingsplan for udbredelsen af telemedicin – kort fortalt\textquotedbl \cite{Regeringen2012} findes 4 overorndet fordele:

\begin{itemize} 
	\item Bedre og mere sammenhængende patientforløb
		\begin{itemize} 

			\item Tættere koordinering mellem den praktiserende læge, den kommunale pleje og sygehusene giver øget kvalitet og sikkerhed i behandlingen.

		\end{itemize} 

	\item Mere individuelt tilrettelagt behandling og selvhjulpne patienter
		\begin{itemize} 

			\item Patienten får indsigt i sin sygdom og bedre muligheder for at deltage aktivt i sin behandling.

		\end{itemize} 

	\item Faglig udvikling af medarbejderne på tværs af sektorer
		\begin{itemize} 

			\item Nye måder at løse opgaverne og faglig sparring med eksperter i andre dele af sundhedsvæsnet giver medarbejderne styrkede kompetencer og mere interssante job.

		\end{itemize} 

	\item Økonomiske gevinster i kommuner og regioner
		\begin{itemize} 

			\item Mere fleksible og effektive måder at organisere arbejdet i sundhedsvæsenet giver færre sygehusindlæggelser, ambulante kontroller og hjemmeplejebesøg.

		\end{itemize} 	
\end{itemize} 




\subsubsection{Erfaring}


	\textquotedbl Klinisk intergret hjemmemonitorering(KIH)-projektet viste, at hjemmemonitorering reducerer ambulante besøg for gravide uden komplikationer og indlæggelsesdage for gravide med komplikationer. For patienter med KOL, inflammatoriske tarmsygdomme og diabetes ændredes antallet af indlæggelsesdage ikke mærkbart. Hospitalsmedarbejdernes ressourceforbrug på tværs af de fem patientgrupper varierede\textquotedbl \cite{Lee2015} 


	\textquotedbl Medarbejderne oplevede, at behandlingskvaliteten højnedes, og de oplevede ligeledes en bedre selvoplevet viden og handlekompetence ved telemedicinsk behandling. Projektet viste derimod ikke en entydig og tydelig ressourcebesparelse ved anvendelsen af telemedicin. Alligevel ser såvel sundhedsfaglige som borgere et stort potentiale i telemedicin.\textquotedbl \cite{Lee2015}


Klinisk Integreret Hjemmemonitorering har teknologisk bidraget med en “KIH Databasen” til opsamling af telemedicinsk data på et nationalt plan således sundhedsfaglige og borgere kan tilgå dataen. 


Derudover udvikles “OpenTele” der som standardkomponent til dataudveksling med eksterne telemedicinske systemer. OpenTele systemet er en open source platform, som består af en mobil og webbaseret sundhedsportal.

 