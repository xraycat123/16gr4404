\section{Telemedicin - intro}

\subsection{Fordele}
Telemedicin kan have en række potentielle fordele både for patienten, økonomien og den sundhedsfaglige medarbejder. Ifølge \textquotedbl National handlingsplan for udbredelsen af telemedicin – kort fortalt\textquotedbl \cite{Regeringen2012} findes 4 overorndet fordele:

\begin{itemize} 
	\item Bedre og mere sammenhængende patientforløb
		\begin{itemize} 

			\item Tættere koordinering mellem den praktiserende læge, den kommunale pleje og sygehusene giver øget kvalitet og sikkerhed i behandlingen.

		\end{itemize} 

	\item Mere individuelt tilrettelagt behandling og selvhjulpne patienter
		\begin{itemize} 

			\item Patienten får indsigt i sin sygdom og bedre muligheder for at deltage aktivt i sin behandling.

		\end{itemize} 

	\item Faglig udvikling af medarbejderne på tværs af sektorer
		\begin{itemize} 

			\item Nye måder at løse opgaverne og faglig sparring med eksperter i andre dele af sundhedsvæsnet giver medarbejderne styrkede kompetencer og mere interssante job.

		\end{itemize} 

	\item Økonomiske gevinster i kommuner og regioner
		\begin{itemize} 

			\item Mere fleksible og effektive måder at organisere arbejdet i sundhedsvæsenet giver færre sygehusindlæggelser, ambulante kontroller og hjemmeplejebesøg.

		\end{itemize} 	
\end{itemize} 

\subsection{Dansk erfaring}
	National handlingsplan for udbredelsen af telemedicin – kort fortalt
 