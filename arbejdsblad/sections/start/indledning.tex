\chapter{Indledning} \label{Indledning}

\textquotedbl Hjertekarsygdom\textquotedbl er en paraplybetegnelse for en bred gruppe af sygdomme, der alle er relateret til det kardiovaskulære system. Denne sygdomsgruppe udgør en af de største sygdomsgrupper i Danmark \cite{livet}. Ved hjertekarsygdomme forsås: Iskæmisk hjertesygdom, Akut myokardieinfarkt, Hjerteklap releteret, Hjertesvigt, Forkammerflimren, Karsygdomme i hjernen\cite{2011}. Af disse registreres årligt omtrent 46.000 nye tilfælde af hjertekarsygdomme (tal fra 2011). Den patientgruppe der har størst risiko for at pådrage sig sygdom relateret til hjertet er mænd over 50 år \cite{2011}, og derudover er der en betragtelig variation blandt forskellige socialgrupper, etniciteter og andre grupper i samfundet \cite{hjerteforening}. I 2011 var antallet af personer der levede med en hjertekarsygdom estimeret til 430.000, og samme år døde over 13.000 personer af hjertekarsygdome som den primære årsag, hvilket svarer til en fjerdedel af alle registrerede dødsfald blandt personer på 35 år eller derover. I rapporten inkluderes en patient først som et nyt tilfælde hvis incidence tilfældet er over 20 år imellem.\cite{2011} Behandling samt medicinering af patienter med hjertekarsygdom i Danmark, resulterer årligt i en samlet udgift på ca. 6,8 milliarder (estimat fra 2011) \cite{hjerteforening}. Der findes flere forskellige metoder og tests til at undersøge det kardiovaskulære system. I 2010 blev der på nationalt plan indført den såkaldte Hjertepakke, der har til formål at standardiserer udredningen i forbindelse med hjerte lidelser/cite{Rolf}. Det har resulteret i at ekokardiogrammet er blivet en central det af udredningerne, da det er et krag at alle patienter skal have for taget et ekokardiogram. /cite{Rolf} På trods af standardiseringen er lægens vurdering og skøn stadig en central det af udredningen./cite{Rolf}
>>>>>>> origin/master


 En af de mindre udbredte metoder kaldes for seismocardiografi (SCG) og giver information om hjertets mekaniske funktion ved at måle hjertets vibrationer på brystvæggen. Anvendelsen af SCG kan spores tilbage til to russiske forskningsprojekter der blev udført  i 1961 og 1964. Ideén var at at måle bevægelse forårsaget af hjertet, der minder om en seismografisk registrering af underjordiske vibrationer for at forudsige jordskælv. Op igennem det 20. århundrede er der opstået en stigende interesse for SCG, og i 1991 blev det for første gang anvendt i kliniske studier. Dog var instrumentation i denne tid alt for besværlig, hvorfor SCG blev stort set opgivet af det medicinske samfund. I det sidste årti har der dog været store teknologiske fremskridt, der har gjort måling og analyse af SCG signaler nemmere, og har åbnet for nye muligheder i klinisk anvendelse \cite{onan} \cite{zanetti}. Ved en standardisering af den kliniske udredning vil en metode der giver flere informationer samtidig være en fordel i form af tidsbesparelse. /fxnote{Dette er muligvis for konkluderende}

\section{Initierende problemstilling} Kan SCG bidrage til de nuværende metoder med yderligere information om hjertets?